\documentclass[10pt,a4paper]{article}

\usepackage{amsmath}
\usepackage{amsfonts}
\usepackage{amssymb}
\usepackage{booktabs}
\usepackage[margin=1.5cm]{geometry}

\author{B00902031 Kevin Tsai, B00902107 Suhorng Yoooooooooooooooo}
\title{CN HW2 report}

\begin{document}

\maketitle

\section{Usage}
    ``make'' compiles two binary files: ``sender'' and ``receiver''. Start receiver at the target computer, and then execute sender to send the file. The file sent will be saved at the working directory of sender, which is normally the directory sender is in. Filename and file permissions are preserved during the transmission. If there already exists a file with the same name, the transmission will be terminated. Sender closes after the transmission.

\section{Package Format}
    The size of every package varies from $9$ bytes to $256$ bytes. The first byte of the package equals the size of the package$ - 1$, that is, $8$ bytes $+$ the size of the content. After this byte is a $4$-byte unsigned integer sequence number followed by a $4$-byte unsigned int crc32 checksum. The rest are the actual data that we want to transfer, whose length can be from $1$ byte to $247$ bytes.

\begin{center}
    \begin{tabular}{lccc}
        Entry name & Size(bytes) & Type & Meaning\\
            \hline
        Package size & $1$ & unsigned char &  size of the package (this entry not counted)\\ 
        Sequence number & $4$ & unsigned integer & sequence number of the package or ACK\\
        Checksum & $4$ & unsigned integer & the crc32 checksum of the whole package when this entry is zero\\
        Data & $1 \sim 247$ & anytype & The data that we want to transfer\\
    \end{tabular}
\end{center}

    When the package is an ACK, an ACK char is put in the data field, so the data field is never empty.

\section{Receiver and Sender FSM}

\section{Loss and error solving}
    \subsection{Package loss solving}
        Nice question.
    \subsection{Package error solving}
        There is a checksum in every package sent. For every package received, the program checks if the checksum in the package is the same as calculated after transmission. If the two values differ, the package is corrupted and hence ignored.
	
\section{Extra work}
    %% maybe FIN is extra work?

\end{document}